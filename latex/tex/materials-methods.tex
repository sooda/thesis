%Tutkimusaineisto ja -menetelmät
\section{Materials and methods}

\subsection{Software tools}

\subsubsection{Programs}

- kinect
- autodesk 123d catch
- meshlab
- meshmixer
- photofly
- sfm toolkit

\subsubsection{Libraries}

- OpenCV
- PCL
- KLT: http://www.inf.ethz.ch/personal/chzach/opensource.html
- http://slowmovideo.granjow.net/


\subsection{Hardware construction}

\subsubsection{Frame}

Aluminium profile system as a frame

Generic camera support screws, 360 angle ball joints?

Adapters for machine vision cameras

Arduino-like adapter HW for sync signals

Connectors, wire

\subsubsection{Cameras}

GigE/USB3/???

- size
- resolution
- speed
- cmos/ccd
- configurability
- noise
- price
- availability

\subsection{Practicalities}

- lens distortion?
- baseline width, focus, depth etc
- compression artifacts are nasty (edge detectors go wild etc.)


\subsection{Reconstruction}

Both shape and texture are considered in this work. Only diffuse color (albedo) is of interest; more complex material properties are assumed to be captured in other means and not spatially varying.

Basic uv mapping. Project texture to computed mesh. Somehow use colors and optical flow everywhere...

Postprocessing: remodel the mesh (face), see what it would look like. Refine parameters to get a similar output as in the photos (normal map etc.), backproject. Use colors and highpass them; assume uniform lightning and locally uniform texture color (bradley). (Simply a rendering technique, that level of detail in 3D structure might not be needed).

