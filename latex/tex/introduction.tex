% Johdanto
\section{Introduction}

%(wikipedia: In recent decades, there is an important demand for 3D content for computer graphics, virtual reality and communication, triggering a change in emphasis for the requirements. Many existing systems for constructing 3D models are built around specialized hardware (e.g. stereo rigs) resulting in a high cost, which cannot satisfy the requirement of its new applications. This gap stimulates the use of digital imaging facilities (like a camera). Moore's law also tells us that more work can be done in software. An early method was proposed by Tomasi and Kanade.[1] They used an affine factorization approach to extract 3D from images sequences. However, the assumption of orthographic projection is a significant limitation of this system.)

% Performance-Driven Facial Animation, Lance Williams, Computer Graphics, Volume 24, Number 4, August 1990
%%http://en.wikipedia.org/wiki/Facial_motion_capture

%% Leave first page empty
% (wat?)
\thispagestyle{empty}

\subsection{Background and motivation}

% content needed; introduce computer graphics

In \emph{computer graphics}, a great interest is the digital synthesis of visual content.
As the processing power of graphics hardware increases, content of increasing detail and quality is attaining popularity, because it can be rendered quickly.
Constructing realistic three-dimensional (3D) structures by hand is difficult and time-consuming, though.
Scanning subjects in real life for automatically building 3D content is becoming increasingly popular.

% where is used

High-detail scanned 3D content has numerous applications in video games, motion picture industry, medical sciences, architectural engineering, archaeology, and more.
Video games with animated, authentic faces of real persons are probably some of the most familiar applications to the general public.
Precise measurements based on reconstructed 3D models are increasingly important landscape mapping and in 3D printed body parts.

% what is 3d content, technically; connection to computer vision

The structure of 3D content is universally described as surfaces consisting of connected polygons, mapped with pictures to add high-detailed color.
While computer graphics synthetises images by drawing millions of polygons to attain a realistic 3D image, \emph{computer vision} tries to attain the inverse:
to study images computationally for understanding the visual information.
Among the applications of computer vision, reconstructing a 3D structure from two-dimensional images is an important one.
3D reconstruction naturally unites computer vision and computer graphics.

% introduce computer vision more and bring mvs in

Computer vision is a large and mature field; the area for acquiring detailed three-dimensional structure from pictures is well known, and the increasing computing power enables the methods to evolve and to even move to mobile devices, making multi-view 3D reconstruction a hot topic.

Multi-view stereo (MVS) is a method for extracting the geometry and color information of a subject, given several images taken from different views.
It can be applied to an unordered collection of pictures or to the imagery of a more strict set of calibrated cameras.
A MVS scanning rig is a specially constructed machinery, consisting of two or more cameras and software, for shooting photos from different viewpoints for computer vision processing.

Plain stereo vision based on fixed two-camera views has been in use in many applications varying from the robotics industry to 3D movie recording.
Stereo vision compares the views of two cameras for depth cues, using similar principles as in the human binocular vision.
Multi-view stereo incorporates a multitude of cameras to acquire a more wide field of view at the same time.
The process of acquiring multi-view stereo images and processing them for 3D data is called 3D scanning or 3D reconstruction.

%The popularity of handheld computers - smartphones - equipped with relatively good quality cameras is currently an attractive target for 3d scanning software, while the actual reconstruction works in the cloud.
%Kinect, the active infrared-based scanner for the xbox console, has reached a phase where it can be considered ubiquitous in gaming.
%Movie and video game studios have used the technology already for years.
%%Current advances in camera technology and price make it possible for individual consumers to do pretty sophisticated reconstruction that has been possible only with a large budget of e.g. a movie studio.
%Even software that uses only a smart phone camera and runs on the same phone has been published.

% static/dynamic definition, surface vs. motion capture, texture

The needs of dynamic content divides 3D scanning in two parts.
Static subjects, i.e.\ stationary rigid bodies, present a simpler problem.
Dynamic (time-varying) subjects move or deform over time, such as human bodies, skin, and clothing.

Static capture is simpler to accomplish, as the time when the pictures are taken does not affect the outcome; even a single moving camera can be used.
Dynamic motion capture has long been used in special effects for movies, for capturing a human body performance with a large set of static cameras, encoding the positions of separate limbs over time.
Capturing the movement in much higher resolution to also scan the whole surface movement and deformation is a more recent interest requiring much more computational power.

While dynamic scanning has interest in scanning faces and human motion for entertainment, static cases have received attention in medical applications, architecture engineering and mapping, crime scene investigation, and so on.

%(Other applications (uses). Movies. Remedy. Medical [essential physics of, bushberg]!).
%Landscape/architecture engineering. Crime scenes (police investigation). Topographic / terrain mapping. Geology, archaeology. Object replication with 3d printing. Aerial photography (digital elevation models DEM)

% XXX nää johki alemmas
%Large scale static 3D scanning and coloring of archeologic sceneries can be done with combining laser scanning and photographs \cite{lerma2010terrestrial}.
%Outside scenes need special care \cite{vu2012high}.
%Small static objects have been scanned successfully with a turntable \cite{fitzgibbon1998automatic}, and structure from motion techniques have recently been presented that impressively recover a structure from pictures taken all over an outdoor location with no a priori information about the camera configurations. \cite{goesele2007multi,furukawa2010towards}.

% on motion capture

Human \emph{motion capture} is one application of dynamic and sparse 3D reconstruction.
Traditional ways to do motion capture use a set of reflective markers attached on a special suit to recover the motion of body parts over time, often encoded as parameters of skeletal joint angles.
The motion information is then reinterpreted for a 3D model to move its parts virtually.
\emph{Surface capture} refers to a more detailed way to recover the motion of a complete surface, such as a human face or a cloth, possibly augmented on a skeletal motion capture data.
Advances in camera resolution make it possible to use only surface texture fiducials without separate markers.

%Reconstructing skin and cloth motion is another highly nonrigid application that can be scanned for acquisition of model parameters. \cite{pritchard2003cloth}

% http://www.siggraph.org/education/materials/HyperGraph/animation/character_animation/motion_capture/history1.htm

% starting our interests; humans in movies+games

%The Matrix \cite{wachowski99matrix}, a movie famous of its bullet-time scenes, used heavy optical flow processing to ``slow time down'';
%then, The Matrix Reloaded \cite{wachowski03reloaded} used a technique called Universal Capture by Borshukov et al. \cite{borshukov05universal} to capture the skin surface of a human head, in order to replicate the head in the form of computer models.

Recent movies and video games (such as \cite{rockstar2011noire}) have shown that by capturing real motion of facial expressions instead of simulating them produces significant results.
Literature on capture of human faces and skin surface motions is large. \cite{deng2007computer}
Professional quality capture still needs a laboratory environment with a large number of cameras \cite{winder2008technical,motionscan}.

% XXX this stuff to some application section somewhere next to theory, too detailed in intro
%The movie industry usually knows the structure of the objects that are tracked; a pre-recorded mesh is used as a helping model (also called ``virtual bones'' to track the object pose. \cite{todo} % universal capture
%It is obviously easier to map image features to a priori information than to recover a fully unknown structure.
%(A pre-recorded higher resolution texture could be used for pore-level details, then tracking with a good-enough resolution and possibly markers to get wrinkles.)

% Realtime capture can be sped up by modeling the face in a parametric space, and mapping video frames to the model.

%The reconstruction and rendering involves lots of filtering of the raw data and further using high-resolution textures to extract better resolution depth data than what is available with computer vision technologies only.

% XXX less background, more contents and motivation

%Motivation: such uses, very popular, much algorithms, need own scanner wow.

\subsection{Goal and scope of the thesis}

% should the face stuff be moved under here? probably parts of it?

%The objective of this work is to describe a complete 3D object scanning and reconstruction pipeline for both shape and texture information.
To construct a 3D scanning rig, the theory behind it is important to consider for justified design choices.
The key hardware for a such rig consists of properly placed cameras, light sources, and a computer.
Putting together all the parts for a general-purpose system needs studying of the end product as a whole.

This thesis is divided in two parts.
First, the relevant theoretical background on image acquisition and multi-view stereo reconstruction is presented.
After that, a more practical part follows: an image acquisition system is built for a part of an end-to-end pipeline for scanned 3D content generation.

The research problem of ``scanning an object in 3D'' is formulated in such a way that a proper hardware can be selected, and the quality of the built system can be addressed.
Performance of the reconstruction algorithms depends on the quality of their input.
How an image is formed with a digital camera, what cameras are available in the market, and how a good input for a reconstruction algorithm can be achieved?
%Reconstruction implementations base their work on the same principles on calibration and triangulation.
To further assess and understand the resulting quality of the reconstruction, the fundamentals of the reconstruction algorithms are presented.

A digital camera based 3D scanning and reconstruction rig is constructed, supported by software tools for acquiring both static image data and video files.
By automating the camera synchronization and image acquisition, the output of the scanner can be fed directly to a reconstruction software.
Final result produced by the rig is a set of images or video, ready to be processed by a reconstruction software for producing a 3D mesh with detailed colors, possibly animated over time.
Feasible readily available programs for reconstructing 3D geometry and texture based on the captured image data are surveyed and presented, for producing textured 3D meshes and point clouds for further handling.
Basic background techniques for taking also motion into account are presented, but the use of state-of-the-art algorithms is left for future implementations.

%The subject is approached from the ground up without assuming too much computer vision knowledge from the reader.
%Some background in computer graphics is expected, but not required.
%%It is assumed that the reader is familiar with some basic concepts of light, such as wavelength and colors. Technical details of cameras and imaging are described and how images are represented digitally.

Aim of the system is in hardware robustness, ease of use and general extendability.
Because there are no detailed plans on all further use of the machinery, it is built to support many kinds of studies in the future without requiring the user to know all implementation issues in detail, and documenting the process in the form of this thesis.

A special case subject in this work is human face: an interesting target because of its complex surface material and diverse ability to produce many different expressions.

%In addition to multiple cameras at different locations and poses, an uniform lightning is also required to minimize specular difficulties in the texture.

The thesis is organised in eight sections, of which this introduction is the first.
Section 2 presents the theoretical background on the elementary methods in image acquisition using a digital camera and the issues in video recording.
Relevant 3D reconstruction theory is described in section 3, with focus on the basics that recent state-of-the-art builds on.
Section 4 extends the 3D topic by introducing issues on motion capture and reconstructing dynamic subjects.
In section 5, the implementation of the 3D scanning rig is described.
Sections 6 and 7 show test experiments on the rig and their results.
Section 8 discusses on future work and concludes the thesis.
