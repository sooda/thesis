\keywords{multi-view stereo, reconstruction, 3D scanning, computer graphics, imaging}
\begin{abstractpage}[english]
In applications of three-dimensional computer graphics, realistic content is increasingly needed.
Rendering of photorealistic models has become possible thanks to computing power, but creating the models is challenging to do by hand, as they need high level of detail in geometry, colors, and animation.
This task is replaced by recovering content from photographs that describe the modeled subject from multiple viewpoints.
In this thesis, elementary theory behind image acquisition and photograph-based reconstruction of geometry and colors is studied.
Recent research and programs for content reconstruction are reviewed.
Based on the studied background, a rig is built using a grid of nine off-the-shelf digital cameras, a custom built remote shutter trigger, and supporting software.
The mission of the rig is to capture quality photographs and video in a reliable and practical manner for processing in multi-view reconstruction programs.

The camera rig is configured to photograph single subjects for multi-view imagery that is processed directly for geometry and color, with little or no image processing done after automatically capturing the imagery.
The rig is proved to work for the purpose of multi-view reconstruction by testing several subjects.
Two typical multi-view stereo reconstruction software pipelines are described and reviewed.
The rig is shown to perform well with little subject-specific tuning;
issues in the setup are described and further improvements are suggested based on the captured test models.
Pure image quality of the cameras was found to be excellent, and most problems arose from uneven lighting and practical issues.
The developed rig was found to produce sub-millimeter scale accuracy in geometry and texture of subjects such as human faces.
Further studies are suggested for lighting, video synchronization and state-of-the-art image processing and reconstruction algorithms.
\end{abstractpage}

\newpage

\keywords{}
\begin{abstractpage}[finnish]
	TODO translate when ready
\end{abstractpage}
