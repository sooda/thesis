\documentclass{standalone}
\usepackage{tikz}
\usetikzlibrary{calc,intersections,through,backgrounds,positioning}
\begin{document}
\begin{tikzpicture}[scale=0.5]

	\tikzset{%
		dimen/.style={<->,>=latex,thin,every rectangle node/.style={fill=white,midway,font=\sffamily}},
	}

	\def\h{10}
	\def\planey{4} % image plane distance
	\def\planew{6} % half of image plane width

	\draw[fill] (0, \h) circle [radius=0.3];
	\node at (0, \h+1) {$\vec X$};


	% origins, T
	% TODO: circles, node ends not exactly at those points
	\draw [dimen] (-9.5,0) -- (9.5, 0) node {$T$};
	\draw [dimen] (0, 0.5) -- (0, \h-0.2) node {$Z$};
	\draw[fill] (-10, 0) circle [radius=0.3];
	\draw[fill] ( 10, 0) circle [radius=0.3];
	\node at (-10, -1) {$\vec C_l$};
	\node at (10, -1) {$\vec C_r$};

	% headings
	\draw [->] (-10, 0) -- (-10, 10);
	\draw [->] (10, 0) -- (10, 10);

	% C-to-P
	\draw [name path=leftline] (-10, 0) -- (0, \h);
	\draw [name path=rightline] (10, 0) -- (0, \h);


	% image planes
	\draw[ultra thick, name path=sensorleft] (-10-\planew, \planey) -- ++(2*\planew, 0);
	\draw[ultra thick, name path=sensorright] (10-\planew, \planey) -- ++(2*\planew, 0);

	\draw [dimen] (-10+\planew, 0.0) -- ++(0, \planey) node {$f$};


	% intersection points at principals and xs
	\path [name intersections={of=leftline and sensorleft,name=lefthit}];
	\path [name intersections={of=rightline and sensorright,name=righthit}];
	\draw[fill] (-10, 4) circle [radius=0.3];
	\draw[fill] (10, 4) circle [radius=0.3];

	\node at (-10.5, 3) {$\vec c_l$};
	\node at (10.5, 3) {$\vec c_r$};

	%\node at (-10+2, \planey+1) {$x_l$};
	\draw [dimen] (-10, \planey+1) -- ($(lefthit-1)+(0,1)$) node {$x_l$};
	\draw [dimen] (10, \planey+1) -- ($(righthit-1)+(0,1)$) node {$-x_r$};
	%\node at (10-2, \planey+1) {$x_r$};

	% p
	\draw[fill] (lefthit-1) circle [radius=0.3];
	\node[below=0.2cm of lefthit-1] {$\vec p_l$};
	\draw[fill] (righthit-1) circle [radius=0.3];
	\node[below=0.2cm of righthit-1] {$\vec p_r$};


\end{tikzpicture}
\end{document}
