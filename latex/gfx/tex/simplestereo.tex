\documentclass{standalone}
\usepackage{tikz}
\begin{document}
\begin{tikzpicture}[scale=0.3]

	\tikzset{%
		dimen/.style={<->,>=latex,thin,every rectangle node/.style={fill=white,midway,font=\sffamily}},
	}

	\draw[fill] (0, 20) circle [radius=0.3];
	\node at (0, 21) {$X$};


	% origins, T
	% TODO: circles, node ends not exactly at those points
	\draw [dimen] (-9.5,0) -- (9.5, 0) node {$T$};
	\draw [dimen] (0, 0.5) -- (0, 19.5) node {$Z$};
	\draw[fill] (-10, 0) circle [radius=0.3];
	\draw[fill] ( 10, 0) circle [radius=0.3];
	\node at (-10, -1) {$C_l$};
	\node at (10, -1) {$C_r$};

	% headings
	\draw [->] (-10, 0) -- (-10, 10);
	\draw [->] (10, 0) -- (10, 10);

	% image planes, at y=4
	\draw[thick] (-13, 4) -- (-7, 4);
	\draw[thick] (13, 4) -- (7, 4);

	\draw [dimen] (-6, 0.0) -- (-6, 4.0) node {$f$};


	% intersection points at principals and xs
	\draw[fill] (-10, 4) circle [radius=0.3];
	\draw[fill] (10, 4) circle [radius=0.3];

	\node at (-10.5, 3) {$c_l$};
	\node at (10.5, 3) {$c_r$};

	\node at (-9, 5) {$x_l$};
	\node at (9, 5) {$x_r$};


	% C-to-P
	\draw (-10, 0) -- (0, 20);
	\draw (10, 0) -- (0, 20);


	% p
	\draw[fill] (8, 4) circle [radius=0.3];
	\node at (8, 3) {$p_r$};
	\draw[fill] (-8, 4) circle [radius=0.3];
	\node at (-8, 3) {$p_l$};


\end{tikzpicture}
\end{document}
