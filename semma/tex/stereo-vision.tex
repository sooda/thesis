\section{Stereo vision}

This chapter introduces the methods for seeing three-dimensional structures from stereo setups, consisting of two or more cameras.
The configuration of a proper setup with two is described and it is extended to the multi-view domain.

\subsection{Camera calibration}

In order to accurately measure a scene with a camera, the camera's properties must be known. Calibrating a camera refers to the act of examine its internal and external configuration in order to map its data to a known coordinate frame.

Calibration is often specified with a camera projection matrix, or several separate matrices.
It may be convenient to store intrinsics and extrinsics separately if the intrinsic matrix is constant for several pictures, for example.

Triangulation or reconstruction of the scene structure given by image pair(s) is usually done on the base of a known relationship between the cameras.
Such relationship, known as calibrating the cameras, can be automatically determined, given by X corresponding points that can be distinguished in each image and matched [?].
Commonly the points are particular \emph{features}, commonly very noticiable edges or corners, found with an algorithm such as SIFT [?], SURF [?] or Harris corner detector [?].

Automatic calibration tools rely on an amount of feature pairs of which the best matches are found, or a known pattern, such as a planar checkerboard pattern [chuan; zhang] whose features are also distinguished with a similar algorithm but a priori knowledge of the object structure is used for precise calibration.
These usually need several pictures taken with the same camera from different poses.

The checkerboard calibration step can also measure optical distortion at the same time; e.g. opencv[?], matlab camera calib toolbox [?]

TODO Figure: show extrinsic in matlab cam calibs, nice pics (both cam and world centered)

Single three-dimensional calibration object is also sufficient blbl

One possible way is direct linear transform (DLT) [hartley/zisserman multiviewgeom]: Solve whole projection matrix instead of individual parameters, decompose projection to intrinsics and extrinsics. For each projected point $\vec x$ from 3d point $\vec X$,

\[
	\vec x = M \vec X
\]

and construct N of these equations and solve for $M$.

Methods that dig the matrix out of a single image have certain restrictions, and won't work if e.g. seven points lie on the same plane [longuet-higgins etc.]

XXX see below. Intrinsic, extrinsic. Distortions. Projection matrices. Camera resectioning.

many single planar chessboard pics vs. a single image of an accurate 3d model.

The scale of values in the equations above affects the precision [hartley, in defense of .., h,ziss]. A similarity transform can be used to modify the values to a more consistent range; this is called normalization of the data.

\subsection{Optical flow}

Long history; several mature commercial video editing products. The Matrix.

Uses: Frame time offset compensation by interpolation (morphing), needs features, direction vector estimation

\subsection{Static capture}

TODO: combine and structure better the following flattened section list

The cameras that are used in capturing a scene can be fixed or positioned arbitrarily; in fact, the \texttt{structure from motion} technique \ref{sec:sfm} enables to use just one camera that is moved around. Accurate and fast reconstructions are still traditionally done with stereo camera pairs, though.

This chapter goes first to the basic ideas on how a camera transforms world space points to a two-dimensional plane and how the camera parameters are usually encoded.

\subsubsection{Coordinate systems and transforms}

Homogenous point description here, [dubrofsky] homography estimation is nice, also refer hartley/zisserman

In computer graphics and vision, points and directions are usually described in homogenous coordinates. Translation, perspective projection, rotation, and other operations are conveniently described by using an additional dimension, of which usually the last element is 1: $(x, y, z, 1)$. All points $(xw, yw, zw, w)$ map to the same point $(x, y, z)$ something something.

Homography definition (mapping of points and lines in $P^2$)

The imaging process essentially captures a projection to a flat two-dimensional plane of the camera's view, as described in section \ref{sec:imaging}.
When relating points between different cameras that view the same scene, their relational positions and rotations must be known.
One of the cameras is often conveniently chosen as the origin of a global coordinate frame, so that its extrinsic parameters become unity transforms (e.g. opencv stereo).
Each three-dimensional point in the world is transformed to the small sensor or film inside the camera, which is then digitized to a discrete two-dimensional grid of pixels. The size of this pixel array (i.e. image) is referred to as the camera's resolution.
Figure \ref{fig:TODO} illustrates this transformation chain, which is encoded as the following equations, given a homogenous point (4-dimensional vector) $X$ representing a 3D location described in physical (e.g. metric) coordinates:

\begin{align}
	x_i &= P X\\
	  &= M_2 X_s\\ % X_s on the sensor
	  &= M_2 R T X\\
	  &= M_3 M_4 R T X\\ % R, T camera pose, M_4 to camera sensor, M_3 to pixel coords
\end{align}

$p_i$ 2d pixel in discrete image, $X_s$ on the sensor, $T$ is a simple translation matrix from $\vec t$

The whole projection $P = M_3 M_4 R T$ can be used as-is without decomposing it to separate matrices, unless the individual parameters are needed. As the chain consists of several matrices, some of them are defined only up to scale; the coordinate systems' units can be chosen freely.

The external camera parameters are called the extrinsics: camera coordinate system position and rotation (heading) in the global space.
Camera position sits at the projection center blah.
The internal parameters, intrinsics, encode how the image is formed on the sensor: they consist of focal length, sensor size and principal point:
\begin{equation}
	M =
	\begin{pmatrix}
		m_x & \gamma & u_0\\
		0   &    m_y & v_0\\
		0   &        0 & 1
	\end{pmatrix}

	\begin{pmatrix}
		f & 0 & 0\\
		0 & f & 0\\
		0 & 0 & 1
	\end{pmatrix}
	=
	\begin{pmatrix}
		\alpha_x & \gamma   & u_0\\
		0        & \alpha_y & v_0\\
		0        & 0        & 1
	\end{pmatrix}
\end{equation}

For simplicity, we denote $\alpha_x = m_x f$, $\alpha_y = m_y f$. $(u_0, v_0)$ is the image center (or principal point). For square pixels, $m_x = m_y$, and for a non-skewed sensor, $\gamma = 0$, which is often the case.

Le image. Horizontal planar triangle, lines between camera origins etc. lecture11.pdf.

\subsubsection{Stereo vision, disparity}

Essential, fundamental matrices. Correspondence problem. Rectification, undistortion. Epipolar geometry.

Two cameras: ideal stereo setup, such as human eyes; cameras are parallel, looking straight and shifted only in the x (or y) axis. Calibration assumed to be known.

From similar triangles with a common vertex at TODO, we get

\begin{align}
	\frac{Z}{T} &= \frac{Z-f}{T - x_l + x_r} \\
	&= \frac{Z-f}{T - d}\\
	ZT - Zd &= ZT - fT\\
	Z &= \frac{fT}{d}
\end{align}


Depth inversely proportional to disparity. Algorithms such as those in OpenCV can compute point clouds from disparity images.

Two or more cameras

(Should this and the next ones be under a "methods for multi view stereo" section?)

\subsubsection{Epipolar geometry}

In stereo vision, the same scene of interest is seen by two or more cameras at the same time.
Epipolar geometry encodes the relations between arbitrarily positioned cameras in a standard way so that coordinates of a 3D point seen in several images can be calculated with triangulation.

A point seen by camera A at 3d point P could be anywhere on the line between A's origin and P.
This line is seen as a single point.
From another viewpoint in camera B, this line equals to some line on B's image plane.
The real point must be on that line.
The inverse applies for any point on B and a line on A.
The lines on the image planes are called epipolar lines.


Essential matrix defines how the camera poses differ by the something something points seen by both. $p_l$, $p_r$ 3d points; vectors from camera origins (camera coordinates!) to the same point

\[
	p_l^T E p_l = 0
\]

Le image. lecture11.pdf. O->p dot (O->O' cross O'->p') = 0

Cross product expressed in a skew-symmetric matrix form is
\begin{equation}
\vec a \times \vec b =
\begin{pmatrix}
	 0   & -a_z &  a_y\\
	 a_z &  0   & -a_x\\
	-a_y &  a_x & 0
\end{pmatrix}
\begin{pmatrix}
	b_x\\b_y\\b_z
\end{pmatrix}
= \vec c
\end{equation}


Fundamental matrix relates the corresponding points in stereo images.

Epipole can be interpreted as the location of another camera as seen by other camera.

\subsubsection{Point matching}

Previously, methods for reconstructing three-dimensional location for a point pair were introduced. To reconstruct a whole scene from a full image, pairwise points must be matched.

Matching can be done in sparse or dense mode; sparse matching finds a set of \texttt{features} from each image, and tries to match them. Dense matching runs through each pixel of one image and tries to find the same from another one; from the coordinate differences between the two images, a disparity map is built. The disparities can be directly transformed to depth values.


\subsubsection{Correspondence and rectification}

In order to triangulate a real point from two or more photos, the location of the point in all images must be known.
Given a pixel in one image, what is the corresponding pixel in another image taken from the same scene?

Rectification is a process that simplifies this search problem by restricting the search to a single dimension.
By aligning the cameras such that their images are coplanar, the search only has to be performed on a line that is parallel to the line connecting the camera centers.
Usually, the calibration between the cameras is known at this point, and foobar.
After rectification, the corresponding lines are axis-aligned (horizontal or vertical) in both images.



\subsubsection{Multi-view stereo}

\subsubsection{Structure from motion}

Structure from motion (SfM) refers usually to recovering the structure of a scene from the motion of a single camera.
For each view, the pose of the camera is determined and the scene structure is extended with the new information in the image.
(pollefeys)

\subsubsection{Bundle adjustment}

\subsubsection{Registration}

Combining 3D meshes from multiple viewpoints (cameras/camera pairs). Also e.g. ransac for removing noise. Iterative closest point fitting.

\subsubsection{Reconstruction}

Uv mapping. Manual work. 3d noise removal; ignore points that have no close pair in other clouds.

Rendering: "as usual".

Postprocessing: remodel the mesh (face), see what it would look like. Refine parameters to get a similar output as in the photos (normal map etc.), backproject. Use colors and highpass them; assume uniform lightning and locally uniform texture color (bradley). (Simply a rendering technique, that level of detail in 3D structure might not be needed). Still, structured light and/or shading assumptions [shape from single image cues/shading trucco,verri p.225] done too.


\subsubsection{Reprojection errors}

The quality of the reconstruction is measured by reprojecting the 3D points back to the cameras and calculating the distance between the projected and original point.
Minimizing all of the points is very expensive but something something bundle adjustment.

Compare to algebraic, geometric etc.

A common way to handle feature errors is Random Sample Consensus (RANSAC). Random subsets of the sample space is iterated, and samples that do not fit well to a model that is constructed of a smaller set are ignored. The iteration that matches most samples is selected.


\subsection{Multiple baseline stereo}
