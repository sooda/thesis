\section{Applications}

The application domain for reconstructing three-dimensional fields is wide, and expands constantly with the ever increasing computing power.
The popularity of handheld computers - smartphones - equipped with relatively good quality cameras is currently an attractive target for 3d scanning software, while the actual reconstruction works in the cloud.
Kinect, the active infrared-based scanner for the xbox console, has reached a phase where it can be considered ubiquitous in gaming.


Facial surface motion capture (mocap) is currently a standard tool in the movie and video game entertainment industry among the more mature mocap for whole body movement to record performances for replaying them later.
Mocap records movements of a human body or another object that is to be recorded so that the movements can be replayed or analyzed.
Recent movies and video games (such as \cite{rockstar2011noire}) have shown that by capturing real motion of facial expressions instead of simulating them produces impressive results.
Human performance capture is traditionally done on special easily distinguishable markers, typically small retroreflective patterns, that are mapped to a model for playback. Playback then interpolates between the recorded positions. Advances in camera resolution make it possible to use only texture features without separate markers.
% http://www.siggraph.org/education/materials/HyperGraph/animation/character_animation/motion_capture/history1.htm

Large scale 3D scanning and texturing of archeologic sceneries is done with laser scanning and photographs \cite{lerma2010terrestrial}.

Dyn dyn dyy to study cloth simulation what
\cite{pritchard2003cloth}

The Matrix (1999) \cite{wachowski99matrix}, famous of the bullet-time scenes, used heavy optical flow processing to ``slow time down''; it is a good example on what is needed to capture data that changes over time in non-controllable and non-repeatable ways.
The Matrix Reloaded \cite{wachowski03reloaded} used a technique called Universal Capture by Borshukov et al.
\cite{borshukov05universal} to encode and simulate accurate facial movements in 3D.

(Other applications (uses). Movies. Remedy? Medical [essential physics of, bushberg]!). Landscape/architecture engineering. Crime scenes (police investigation). Topographic mapping. Geology, archaeology. Object replication with 3d printing. Aerial photography (digital elevation models DEM)

Small objects have been scanned successfully with a turntable \cite{fitzgibbon1998automatic}, and structure from motion techniques have recently been presented that impressively recover a structure from pictures taken all over an outdoor location with no a priori information about the camera configurations. \cite{goesele2007multi,furukawa2010towards}.

