% Johdanto
\section{Introduction}
%Non-static Object Capture Using Multi-view Stereo Video
%% Leave first page empty
% (wat?)
\thispagestyle{empty}

Computer vision is a mature field; the steps of acquiring three-dimensional structure of a real-life scene are well known.
Multiple view geometry (multi-view stereo) is one method to interpret depth in camera images and thus to acquire 3D scans.
\cite{hartley03multiview,szeliski10vision,trucco1998introductory}
For detailed scanning dealing with arbitrary rigs and reconstruction types, there exist a wide range of algorithms, each suited for different setups, some more general than others.
\cite{seitz2006comparison}.

Stereo vision and the related motion capture are well studied fields.
3D scanning and reconstruction has been successfully done on single objects and bigger scenes accurately with the help of ever increasing computing power,
\cite{goesele2007multi,furukawa2010towards,pollefeys1999hand}
and mocap in coarser form is an ubiquitous tool in the film industry [?].
Recently introduced Microsoft Photosynth [?], for example, is able to take an arbitrary collection of photos from the internet and reconstruct three-dimensional models of publicly photographed targets.
The web-based tool Autodesk 123D Catch [?] allows any hobbyist to upload a group of photographs to an automatic reconstruction engine that generates a three-dimensional model.

Using similar principles as the human eyes to calculate point disparity and depth of individual pixels from photos, 3D point clouds can be constructed, given only a set of two-dimensional images taken of a same target from different poses and and a few camera parameters.
\cite{hartley03multiview}
This technique, called photogrammetry, has applications in many fields ranging from mapping of larger scenes to scanning of individual objects.
The principles are old and mature; only current state of computing power available to individuals has introduced lots of progress in automatic software tools.
Three-dimensional structure can be scanned with other means too, such as laser range finders [?], structured light [?], or computer tomography [?]; in this seminar work, the focus is kept on stereo-based computer vision.

Recent advances in additive fabrication, informally called 3D printing, have brought the need to scan the geometry of arbitrary objects accurately in order to replicate them by printing.
Hobby-grade printer enthusiasts have needed to rely mostly on manual measurements and/or lots of manual work in modeling tools.
Even mobile phones have software for reconstructing an object by simply rotating the phone around it, taking pictures automatically using the phone's camera [?]. Professional quality capture still needs a laboratory environment with a large number of cameras [?,?].

When 3D scanning is extended to take into account temporal changes in geometry or appearance (color), more complex hardware and calculations are needed in order to cope with the changing data.
Non-static cases need a larger set of cameras set up so that the captured target can move while the geometry is imaged from several different directions. Applications include for example performance capture for video games or movies [?], cloth something something [?], geoscience [?] and realtime traffic analysis [?]
A related field in robotics is simultaneous localization and mapping (SLAM), where the surroundings are mostly static but the camera moves in an unknown environment. [?]


For there are several methods on the topic, choices must be made when implementing the reconstruction setup.
Different purposes have pros and cons, and the length of this seminar work cannot cover all the details; therefore, only the most common methods are described, sometimes referring to more detailed papers for details.
The structure of this work is divided as follows in a bottom-up approach: the first part presents the basic geometric principles behind most implementations on stereo vision, starting from grabbing images of real-life scenes, finally extending to multi-view stereo.
The same chapter describes what happens in the dynamic case.
Then, the next chapter focuses on what should be done when tracking the dynamic actions of a scanned target.
Finally, the current state of software tools and applications are presented. The last chapter summarizes the methods.
In-depth mathematical details are kept to minimum, keeping the focus on the steps and principles of the whole pipeline from cameras to triangular mesh output. All the chapters present the additional challenges brought by dynamic targets when applicable.

In this seminar work, the basic technical aspects in the problem of three-dimensional scene reconstruction in multi-view stereo is introduced. The problem is extended in the non-static domain. Current software tools to solve the problem are introduced. The current state of the art and trends in the field are quickly surveyed.
